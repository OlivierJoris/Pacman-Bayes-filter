\documentclass{article}
\usepackage[utf8]{inputenc}

\usepackage{amsfonts}
\usepackage{amssymb}
\usepackage{amsmath}
\usepackage{amsthm}
\usepackage{enumitem}

\usepackage{bbold}
\usepackage{bm}
\usepackage{graphicx}
\usepackage{color}
\usepackage{hyperref}
\usepackage[margin=2.5cm]{geometry}

\begin{document}

% ==============================================================================

\title{\Large{INFO8006: Project 3 - Report}}
\vspace{1cm}
\author{\small{\bf Mr Pacman - s111111} \\ \small{\bf Miss Pacman - s222222}}

\maketitle

% ==============================================================================

\section{Bayes filter}

\begin{enumerate}[label=\alph*.,leftmargin=*]
    \item The observable evidence variables at time \textit{t} are defined in
    this way for the ghost \textit{i} : \\
    
    $noisyDistance(i) = 
    manhattanDistance(ghost(i), Pacman) + Binom(n, p) - n * p$ \\

    where \textit{noisyDistance(i)} corresponds to the observable evidence variable 
    which is tracking the ghost \textit{i} at time \textit{t}, \textit{manhattanDistance(ghost(i), Pacman)}
    corresponds to the Manhattan distance between the ghost \textit{i} and Pacman at time \textit{t},
    \textit{Binom(n, p)} corresponds to a random variate following a binomial distribution with the given parameters
    \textit{n} and \textit{p}, \textit{p} = $\frac{1}{2}$, and n = $\frac{sensor\_var}{p \ (1 - p)} = \frac{sensor\_var}{\frac{1}{4}}$ with \textit{sensor\_var} being the variance of the rusty sensor.
    
    \item The transition model $P_a(X_t | X_{t-1}, g)$, with $a$ being a legal action taken and $g$ being the type of ghost (confused, afraid, scared), is, in general, $P_a(X_t | X_{t-1}, g) = \alpha * \gamma$. In detail :
    \[
  P_a(X_t | X_{t-1}, g)=\begin{cases}
               P_a(X_t | X_{t-1}, g) = \alpha * 1 \text{ if g} = \text{confused} \\
               P_a(X_t | X_{t-1}, g) = \alpha * 2 \text{ if g} = \text{afraid and }d(X_t, P) \geq d(X_{t-1},P) \\
               P_a(X_t | X_{t-1}, g) = \alpha * 8 \text{ if g} = \text{scared and }d(X_t, P) \geq d(X_{t-1},P) \\
               P_a(X_t | X_{t-1}, g) = \alpha * 1 \text{ else}
            \end{cases}
\]
where:
\begin{itemize}
	\item $d(X_t, P)$ is the Manhattan distance between Pacman and the ghost at time t.
	\item $\alpha = \frac{1}{\sum_{i=1}^N {\gamma_i}}$ with $N$ being the number of legal actions.
\end{itemize}

\end{enumerate}

\section{Implementation}

\begin{enumerate}[label=\alph*.,leftmargin=*]
    \item \textbf{\textit{Leave empty.}}
\end{enumerate}

\section{Experiment}

\begin{enumerate}[label=\alph*.,leftmargin=*]
    \item 
    \item The quality of the belief state can be computed by evaluating the difference of probabilies between the real state of the game and the computed belief state.
    The closer this difference is to 0, the better.
    \item
    \item
    \item
    \item
    \item \textbf{\textit{Leave empty.}}
\end{enumerate}

% ==============================================================================

\end{document}
